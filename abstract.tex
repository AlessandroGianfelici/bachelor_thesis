\chapter*{\centering \begin{normalsize}Abstract\end{normalsize}}
\begin{quotation}
\noindent
Questa tesi tratta il problema della violazione di CP nel sistema $P^0$ - $\bar{P}^0$ dove
$P^0$, in base ad una notazione largamente impiegata nella letteratura sull'argomento, può simboleggiare uno qualsiasi dei mesoni neutri $K^0$, $D^0$, $B^0$ o $B_s^0$.

Nella prima parte della tesi è proposto l' inquadramento teorico del fenomeno, definendo i tre operatori di parità ($\mathscr{P}$), coniugazione di carica ($\mathscr{C}$) ed inversione temporale ($\mathscr{T}$)
nell'ambito della meccanica quantistica, quindi definendo il formalismo adatto a descrivere il sistema di mesoni neutri generico $P^0$ - $\bar{P}^0$. 
In seguito sono discusse le proprietà delle interazioni deboli nel quadro del Modello Standard, con una descrizione della matrice CKM e dei sei relativi triangoli di unitarietà. 

Nella seconda parte è proposta una misura dell'angolo $\gamma$ della matrice CKM, realizzata utilizzando i recenti risultati ottenuti dalla Collaborazione LHCb con 
i dati  corrispondenti ad una luminosità integrata di $1$ $fb^{-1}$ raccolti nel 2011 nelle collisioni protone-protone ad una energia di 7 TeV ad LHC.

La misura dell'angolo $\gamma = arg [-V_{ud}V_{ub}^*/(V_{cd}V_{cb}^*)]$ è una delle misure più importanti che possono essere compiute dall'esperimento LHCb.
 
La presente analisi è stata realizzata utilizzando i risultati pubblicati da LHCb per il decadimento  $B^{\pm}\rightarrow DK^{\pm}$ ed utilizzando una procedura 
basata sulla statistica bayesiana. È stato ottenuto il valore: 
\begin{equation*}
 \gamma = (60.3^{+17.1}_{-13.5})^{\circ}
\end{equation*}
Questo risultato è in ottimo accordo con le misure ottenute dalle Collaborazioni BaBar \cite{BaBar} e Belle \cite{Belle}: $\gamma_{BaBar} = (69^{+17}_{-16})^{\circ}$ e $ \gamma_{Belle} =  (68^{+15}_{-14})^{\circ}$.


La precisione raggiunta nella determinazione del valore di $\gamma$ indica che LHCb ha un enorme potenziale sulla misura di questa grandezza, considerando che l'analisi 
condotta in questa tesi è stata realizzata con il $40\%$ della statistica totale raccolta da LHCb ad oggi (LHCb ha infatti raccolto $2.5$ $fb^{-1}$).


\begin{comment}Solo Conclusione
Il metodo teoricamente più pulito per estrarre la fase debole $\gamma$ è quello di misurare osservabili sensibili a $\gamma$ mediante i decadimenti 
$B^{\pm}\rightarrow DK^{\pm}$ e $B^{\pm}\rightarrow D\pi^{\pm}$. 
In questo caso, l'analisi è stata realizzata  è\end{comment}
\end{quotation}
\clearpage