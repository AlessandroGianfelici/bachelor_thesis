\chapter*{Introduzione}
\noindent
Fino ai primi anni cinquanta dello scorso secolo si pensava che ogni legge fisica fosse invariante per trasformazioni di simmetria discrete: parità, coniugazione di carica ed inversione temporale.
In altre parole si riteneva che, scelto un qualsiasi fenomeno fisico, dovesse essere permesso dalle leggi naturali anche il fenomeno ottenuto da questo cambiando il segno delle tre coordinate spaziali (parità), invertendo la freccia del tempo (inversione temporale) e sostituendo ogni particella con la rispettiva antiparticella (coniugazione di carica).

Quando i lavori di Tsung Dao Lee e Chen Ning Yang smentirono in maniera inconfutabile questa visione delle leggi fisiche, dimostrando la violazione della parità da parte dell' interazione debole, si accese nella comunità scientifica un forte interesse verso le simmetrie discrete.
Ben presto ci si rese conto che la forza debole violava anche la coniugazione di carica  in maniera tale, tuttavia, che la simmetria combinata CP sembrasse comunque conservata.

Dal punto di vista teorico la simmetria CP ricopre un ruolo molto importante nel quadro delle simmetrie discrete: a causa del teorema CPT, che impone l'invarianza sotto l'azione combinata delle tre simmetrie discrete di qualsiasi processo fisico  Lorentz-invariante, una violazione di CP  implica necessariamente una violazione di T di pari entità. 

A prima vista potrebbe sembrare che la distinzione tra passato e futuro dei processi fisici sia, di per sé, una realtà autoevidente: raramente i fenomeni macroscopici con i quali siamo familiari presentano un comportamento temporalmente simmetrico, basti pensare a qualsiasi processo diffusivo o termodinamico per il quale la freccia del tempo è orientata inequivocabilmente nella direzione che porta all'aumento dell'entropia.
In realtà, la simmetria per T è qualcosa di più sottile, che riguarda esclusivamente il mondo microscopico. Infatti, l'apparente asimmetria temporale dei processi macroscopici è dovuta semplicemente alla natura statistica degli stessi. 

Il fatto che anche a livello delle particelle elementari esistessero fenomeni in grado di distinguere il passato dal futuro era completamente inaspettato, fino a quando gli esperimenti di  Cronin e Fitch
 dimostrarono la violazione di CP da parte del sistema dei kaoni neutri. Le implicazioni teoriche di questa scoperta sono notevoli, infatti potrebbero essere il primo passo per giungere finalmente alla comprensione dei processi fisici che, negli attimi immediatamente successivi al Big Bang, portarono alla formazione dell'Universo. 

Già nel 1967 il fisico russo A. Sakharov aveva proposto tre condizioni che una reazione fisica elementare avrebbe dovuto soddisfare affinché questa avesse potuto produrre materia ed antimateria in quantità differenti, impedendo dunque che la reciproca annichilazione lasciasse dietro di sé un Universo formato dalla sola radiazione cosmica di fondo. Queste condizioni sono:
\begin{enumerate}
 \item Violazione del numero barionico $B$
 \item Violazione di C e di CP
 \item Possibilità di avvenire fuori dall'equilibrio termico
\end{enumerate}
Benché il Modello Standard ammetta una sorgente di violazione di CP, attraverso un meccanismo ipotizzato nel 1973 dai fisici 
giapponesi Kobayashi e Maskawa, e che finora abbia correttamente spiegato tutte le osservazioni sperimentali, l'entità di violazione prevista è insufficiente 
a spiegare la bariogenesi primordiale. Per questo motivo la ricerca di sorgenti di violazione di CP oltre il Modello Standard è uno dei campi della fisica delle 
alte energie sui quali al momento si concentrano gli sforzi maggiori, sia sul versante teorico che su quello sperimentale. 

Il rilevatore LHCb, del C.E.R.N di Ginevra, è lo strumento più potente oggi disponibile per lo studio della dinamica del \emph{flavour} pesante. 
Lo scopo dell'esperimento è di misurare con precisione i parametri caratteristici della matrice CKM, in particolare la fase debole $\gamma$, oggetto di questa tesi.

Per ottenere una misura di $\gamma$ si devono misurare osservabili dipendenti da $V_{ub}$ mediante le interferenze di ampiezze quantistiche dei decadimenti ad 
albero del $B^+$ e del $B^-$.

I metodi di misura di $\gamma$ sono classificati in base alla natura dei prodotti finali dei decadimenti ad albero dei mesoni $B$. 
La precisione raggiungibile nella misura di $\gamma$ può essere aumentata combinando più metodi indipendenti.
In questo lavoro di tesi ne sono stati impiegati tre: GLW, ADS e GGSZ.

I dati utilizzati per questa analisi sono quelli raccolti da LHCb nel corso del 2011. Il valore ottenuto per $\gamma$ 
raggiunge una precisione di circa $15^{\circ}$, comparabile con i risultati ottenuti
delle collaborazioni BaBar e Belle.

Considerato che l'analisi condotta in questa tesi è stata realizzata con il $40\%$ 
della statistica totale raccolta, LHCb dimostra di possedere un grande potenziale per la misura di $\gamma$ (LHCb ha infatti raccolto 2.5 $fb^{-1}$).

In futuro, quindi, le misure di LHCb potranno portare ad un notevole progresso nella conoscenza dell'interazione debole nel settore dei quark e nella dinamica dei 
\emph{flavour} pesanti, forse arrivando a scoprire fenomeni di Nuova Fisica oltre il Modello Standard.
