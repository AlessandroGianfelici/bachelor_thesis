
    \index{appendici}
  
\chapter{Equazione di Dirac}%SEZIONE COMPLETA
\noindent
L'equazione di Dirac venne introdotta in meccanica quantistica da P.A.M. Dirac per descrivere la dinamica di un fermione di spin $1/2$ in regime relativistico \cite{Dirac}.
In analogia con quanto gi\`{a} fatto prima di lui da Schroedinger, che nel ricavare la sua celebre equazione riscrisse in termini operatoriali l'Hamiltoniana classica, Dirac 
calcol\`{o} l'equivalente in termini di operatori dell'Hamiltoniana relativistica.\\Nella forma in cui viene solitamente scritta, questa risulta essere:
\begin{equation}\label{HAMILTONIANA}
H = \sqrt{m^{2}c^{4}+p^{2}c^{2}}
\end{equation}
Sostituire direttamente le espressioni operatoriali $H=i\hbar\frac{\partial}{\partial t}$ e $\vec{p}=-i\hbar \vec{\nabla}$ nella 
\eqref{HAMILTONIANA} porta ad una equazione contenenti radici quadrate di operatori differenziali, cioè ad oggetti non trattati dall'analisi
matematica. Se invece si cerca di aggirare questa difficoltà elevando al quadrato i due lati della \eqref{HAMILTONIANA} prima
di eseguire le sostituzioni (che è il procedimento seguito da \emph{Klain e Gordon}, che in questo modo dedussero la prima equazione d'onda relativistica)
si ottiene una equazione del secondo ordine nel tempo. Una tale equazione non ammette una densità di corrente di probabilità definita positiva, il che è in contrasto con l'interpretazione probablistica della Meccanica Quantistica.
Dirac dedusse la prima equazione d'onda relativistica che fosse coerente con questa interpretazione e la applicò alla descrizione fisica 
dell'elettrone.
Egli cercò una relazione tra energia ed impulso che fosse lineare e compatibile con la \eqref{HAMILTONIANA}:
\begin{equation}
 H = c\sum_i p_i\gamma^i + \gamma^0 mc^2
\end{equation}
Oppure, ricordando che $H = E$:
\begin{equation} \label{k}
 E - c\sum_i p_i\gamma^i - \gamma^0 mc^2 = 0
\end{equation}
Dove le quattro componenti di $\gamma$, ($\gamma^1$, $\gamma^2$, $\gamma^3$ e $\gamma^0$) sono indipendenti da ${r}$, $t$, 
${p}$ ed $E$.
A questo punto si eseguono le sostituzioni operatoriali e si ottiene l'\emph{equazione di Dirac}:
\begin{equation}
  i\hbar \frac{\partial}{\partial t}\Psi_i = i\hbar c\sum_{j=1}^{N}\sum_{k=1}^{3}\gamma_{ij}^k\frac{\partial}{\partial x_k} \Psi_j +\sum_{j=1}^{N}\gamma_{ij}^0mc^2\Psi_j
\end{equation}
dove $N$ è il rango delle matrici quadrate $\gamma^1$, $\gamma^2$, $\gamma^3$ e $\gamma^0$.
L'Hamiltoniana $H$ deve essere Hermitiana, $H = H^{\dag}$, per questo motivo anche le matrici $\gamma^1$, $\gamma^2$, $\gamma^3$ e $\gamma^0$ devono
esserlo. Moltiplicando ora da sinistra la \eqref{k} per $E + c\sum_i p_i\gamma^i + \gamma^0 mc^2 = 0$ ed interpretando il risultato in termini operatoriali si ottiene:
\begin{equation}\label{pippo}
 \Big\{E^2-c^2\Big[\sum_{k=1}^{3}(\gamma^k)^2p{_k}^2 + \sum_{k<l}(\gamma^k\gamma^l+\gamma^l\gamma^k)p_kp_l\Big] - mc^3\Big[\sum_{k=1}^{3}(\gamma^k\gamma^0+\gamma^0\gamma^k)p_k\Big] - m^2c^4{\gamma^0}^2\Big\}\Psi = 0
\end{equation}
Confrontando la \eqref{pippo} con la relazione relativistica tra energia ed impulso $E^2 = p^2c^2 + m^2c^4$ si ottiene che le due sono coerenti l'una con l'altra
se valgono le seguenti relazioni:
\begin{equation} \label{relazione1}
 (\gamma^1)^2=(\gamma^2)^2=(\gamma^3)^2=(\gamma^0)^2 = I
\end{equation}
\begin{equation} \label{relazione2}
 \{\gamma^1,\gamma^2\} = \{\gamma^2,\gamma^3\} = \{\gamma^3,\gamma^1\} = 0
\end{equation}
\begin{equation} \label{relazione3}
 \{\gamma^1,\gamma^0\} = \{\gamma^2,\gamma^0\} = \{\gamma^3,\gamma^0\} = 0
\end{equation}
Si può dimostrare che il rango minino delle matrici $\gamma^1$, $\gamma^2$, $\gamma^3$ e $\gamma^0$ affinchè le relazioni 
\eqref{relazione1}, \eqref{relazione2} e \eqref{relazione3} siano soddifatte è $N = 4$. Ciò implica che anche la funzione d'onda $\Psi$ che soddisfa il 
sistema deve avere quattro componenti. Ciascuna componente $\Psi_i$è detta \emph{spinore di Dirac}. 
Esistono infinite rappresentazioni per le matrici che entrano nella definizione dell'equazione di Dirac, ma la più utilizzata in fisica è la cosiddetta 
\emph{rappresentazione di Dirac}:
\begin{equation}
 \gamma^0 = \Big(\begin{matrix} I & 0 \\ 0 & -I \end{matrix}\Big)
\end{equation}
\begin{equation}
 \gamma^i = \Big(\begin{matrix} 0 & \sigma_i \\ \sigma_i & 0 \end{matrix}\Big)
\end{equation}

dove $i \in \{1,2,3\}$, $I$ è la matrice identità di rango $2$ e le $\sigma_i$ sono le \emph{matrici di Pauli}:
\begin{equation}\label{sigma1}
 \sigma_1 = \Big(\begin{matrix} 0 & 1 \\ 1 & 0 \end{matrix}\Big)
\end{equation}
\begin{equation}\label{sigma2}
 \sigma_2 = \Big(\begin{matrix} 0 & -i \\ i & 0 \end{matrix}\Big)
\end{equation}
\begin{equation}\label{sigma3}
 \sigma_3 = \Big(\begin{matrix} 1 & 0 \\ 0 & -1 \end{matrix}\Big)
\end{equation}
L'equazione di Dirac, qui dedotta per una particella libera, è facilmente generalizzabile al caso di una particella carica in un campo elettromagnetico
eseguendo le sostituzioni, note dall'elettromagnetismo classico:
\begin{equation}
 p' \rightarrow \Big(p-\frac{qA}{c}\Big)
\end{equation}
\begin{equation}
 E' \rightarrow E - qV
\end{equation}
dove $A$ è il potenziale vettore, $V$ il potenziale scalare e $q$ la carica elettrica della particella in esame.
In questo modo si ottiene l'equazione di Dirac per una particella carica immersa in un campo elettromagnetico:
\begin{equation}
 \Big[\Big(i\hbar\frac{d}{dx^\mu}-\frac{q}{c}A_\mu\Big) \gamma^\mu - mc\Big] \Psi = 0
\end{equation} \cite{Bransden}


\chapter{Formula di Sokhotski–Plemelj}%SEZIONE COMPLETA
\noindent
La \emph{formula di Sokhotski–Plemelj} viene spesso scritta, in maniera simbolica, come:
\begin{equation}
 \lim_{\epsilon \rightarrow 0}\frac{1}{x\pm i\epsilon} = P\Big(\frac{1}{x}\Big) \mp \pi\delta(x)
\end{equation}
Dove il limite è inteso in senso distribuzionale, in $\mathscr{D}'$.
Per dimostrare questa formula, consideriamo la distribuzione di tipo funzione:
\begin{equation}
T = \ln(x\pm i\epsilon) = \ln(z)
\end{equation}
con $z \in \mathbb{C}$. Si sceglie la determinazione principale del logaritmo:
\begin{displaymath}
\left\{
\begin{array}{l}
\ln(z) = \ln|z| + i\arg(z) \\
-\pi<\arg(z)<\pi
\end{array}
\right.
\end{displaymath} 
Quindi, per $x\in\mathbb{R}$, si ha $\arg(x) = 0$ se $x>0$ e $\arg(x) = \pi$ se $x<0$. 
La funzione logaritmo, con $\epsilon$ finito, è localmente sommabile, quindi permette di definire una distribuzione $\mathscr{D}'$.
La derivata distribuzionale coincide con la derivata ordinaria, per cui si può scrivere:
\begin{equation}\label{1}
 \frac{d}{dx}\ln(x\pm i\epsilon) = \frac{1}{x\pm i\epsilon} =  \frac{d}{dx}\ln|x\pm i\epsilon| +i\frac{d}{dx}\arg(x\pm i\epsilon)
\end{equation}
La derivata distribuzionale è un'applicazione continua in $\mathscr{D}'$, per cui vale:
\begin{equation}\label{2}
\mathscr{D}'-\lim_{\epsilon\rightarrow0^+}\arg(x\pm i\epsilon) = \pm\pi\theta(-x) = \pm\pi\mp\pi\theta(x)
\end{equation}
Dove con $\theta(x)$ si è indicata la \emph{funzione di Heaviside}, che assume valore $1$ se il suo argomento è maggiore di $0$ e $0$ altrimenti.
Ricordando infine che, in senso distribuzionale:
\begin{equation}\label{4}
 \frac{d}{dx} \theta(x) = \delta(x) 
\end{equation}
\begin{equation}\label{3}
 \frac{d}{dx} \ln(x) = P\Big(\frac{1}{x}\Big) 
\end{equation}
(dove $P$ indica il valore principale di Cauchy) si ottiene, sostituendo la \eqref{3}, la \eqref{2} e la \eqref{4} nella \eqref{1}:
\begin{equation}
  \lim_{\epsilon \rightarrow 0}\frac{1}{x\pm i\epsilon} = P\Big(\frac{1}{x}\Big) \mp \pi\delta(x)
\end{equation}





