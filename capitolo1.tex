     %%%%%%%%%%%%%%%%%%%%
     %                  %
     %  capitolo1.tex   %
     %                  %
     %%%%%%%%%%%%%%%%%%%%


\chapter{Trasformazioni di simmetria P, C e T}
\noindent
La coniugazione di carica (C), la parità (P) e l'inversione temporale (T) sono  trasformazioni discrete di sistemi fisici.
È appurato che prese singolarmente esse siano simmetrie valide solo parzialmente, in quanto, ad esempio, sono tutte e tre violate nelle interazioni deboli.
La loro composizione CTP è invece ritenuta una simmetria esatta della Natura, sempre soddisfatta da ciascuna interazione fondamentale. 

Quanto appena affermato corrisponde all'enunciato di un importante teorema (\emph{Teorema CTP}), che rappresenta una convinzione profondamente 
radicata nella fisica moderna; tanto che la verifica dell'invarianza per CTP è uno dei criteri per verificare la coerenza di nuove teorie.

Nel seguito si definiranno le tre trasformazioni discrete di simmetria in ambito quantistico, 
e saranno definiti gli opertatori $\mathscr{C}$, $\mathscr{P}$, $\mathscr{T}$ ad esse associati.

%
Definto con  $\mathscr{O}$ uno qualsiasi degli operatori  $\mathscr{C}$, $\mathscr{P}$ o $\mathscr{T}$,  ci si attende che per un sistema simmetrico si abbia:
\begin{equation}
[\mathscr{O},\mathscr{H}] = 0
\end{equation}
%
%Nella meccanica classica l'applicazione di una simmetria discreta
%non influisce sulle coordinate temporali del sistema, cioè
%commuta con la traslazione temporale $t'\rightarrow t + t_0$.
%Applicando il principio di corrispondenza, nel passaggio alla
%meccanica quantistica è lecito attendersi tali trasformazioni
%abbiano analoghe proprietà, cioè che:
%
cioè che ciascuno dei tre operatori commuti con l'Hamiltoniana del sistema. Come detto però ciò non è vero nel caso ci si riferisca alle interazioni deboli. 
%Se ne conclude che in processi (interazioni) non invarianti per una delle
%trasformazioni discrete indicate non \`e possibile definire in maniera
%consistente il relativo operatore. Questa difficolt\`a pu\`o essere
%aggirata limitandosi a definire gli operatori  al caso delle interazioni invarianti:
%elettromagnetiche e forti. 
%Si possono utilizzare gli operatori cos\`i definiti per verificare la violazione di C, P o T nei
%processi deboli.
%
In una condizione in cui $\mathscr{O}$ possa essere considerato una simmetria del sistema, dati due stati fisici $\alpha$ e $\beta$, deve valere la relazione:
\begin{equation}
 |\langle\alpha|\mathscr{O}^{\dag}\mathscr{O}|\beta\rangle|^2=|\langle\alpha|\beta\rangle|^2
\end{equation}
%
cui corrispondono due diverse possibilit\`a:
\begin{displaymath}
\left\{
\begin{array}{l}
 \langle\alpha|\mathscr{O}^{\dag}\mathscr{O}|\beta\rangle = \langle\alpha|\beta\rangle \\
 \langle\alpha|\mathscr{O}^{\dag}\mathscr{O}|\beta\rangle = {\langle\alpha|\beta\rangle}^{*}
\end{array}
\right.
\end{displaymath}
%
pertanto gli operatori delle trasformazioni di simmetria possono essere unitari o antiunitari. 
Nel seguito si dimostrerà che $\mathscr{C}$ e $\mathscr{P}$ sono entrambi unitari, mentre $\mathscr{T}$ \`e un operatore antiunitario.

\section{Parità}
\noindent
La simmetria di parit\`a di un sistema fisico consiste nell'invarianza a seguito alla trasformazione:
\begin{displaymath}
\left\{
\begin{array}{l}
x \rightarrow -x\\
y \rightarrow -y\\
z \rightarrow -z
\end{array}
\right.
\end{displaymath}
Essa è chiamata anche simmetria per riflessione, perch\'e l'inversione degli assi pu\`o essere ottenuta componendo una rotazione di $\pi/2$ con la riflessione dell'asse di rotazione.
%
\subsection{L'operatore $\mathscr{P}$} 
\noindent
Secondo il Principio di Corrispondenza ci si aspetta che il valore di aspettazione dell'operatore posizione $\vec{X}$, cambi di segno in seguito ad una trasformazione di parit\`a:
\begin{equation}
 \langle\psi|\mathscr{P}^{\dag} \vec{X} \mathscr{P}|\psi\rangle= -\langle\psi|\vec{X}|\psi\rangle
\end{equation}
Ci\`o implica che:
\begin{equation}
 \mathscr{P}^{\dag} \vec{X} \mathscr{P} = -\vec{X}
\end{equation}
o, equivalentemente, che $\vec{X}$ anticommuti con $\mathscr{P}$: 
\begin{equation}
\{\vec{X},\mathscr{P}\}=0.
\end{equation}

Un generico autostato delle coordinate $|\vec{x}\rangle$, cui corrisponde l'autovalore $\vec{x}$ si trasformer\`a quindi secondo la relazione:
 \begin{displaymath}
\left\{
\begin{array}{l}
\vec{X}|\vec{x}\rangle = \vec{x}|x\rangle\\
\vec{X}\mathscr{P}|\vec{x}\rangle = -\mathscr{P}\vec{X}|\vec{x}\rangle = -\mathscr{P}\vec{x}|\vec{x}\rangle
\end{array}
\right.
\end{displaymath}
Dunque $\mathscr{P}|\vec{x}\rangle$ è a sua volta un autostato dell'operatore posizione, corrispondente all'autovalore $-\vec{x}$.
Tuttavia, all'autovalore $-\vec{x}$ corrisponde anche l'autostato $|-\vec{x}\rangle$:
 \begin{displaymath}
\left\{
\begin{array}{l}
\vec{X}|-\vec{x}\rangle = -\vec{x}|-x\rangle\\
\vec{X}\mathscr{P}|\vec{x}\rangle = -\vec{x}|-x\rangle\\
\end{array}
\right.
\end{displaymath}
Per cui $\mathscr{P}|\vec{x}\rangle$ e $|-\vec{x}\rangle$ possono differire al più per un fattore di fase arbitrario:
\begin{equation}
\mathscr{P}|\vec{x}\rangle = e^{i\phi}|-\vec{x}\rangle
\end{equation}
Per convenienza, si considererà $\phi = 0$, quindi
$\mathscr{P}|\vec{x}\rangle = |-\vec{x}\rangle$. Applicando due
volte $\mathscr{P}$ si ottiene:
\begin{equation}
\mathscr{P}^2|\vec{x}\rangle = -\mathscr{P}|\vec{x}\rangle = |\vec{x}\rangle
\end{equation}
e quindi:
\begin{equation}
\mathscr{P}^2= 1
\end{equation}
Il che significa che l'operatore parità è hermitiano oltreché unitario:
\begin{equation}
\mathscr{P} = \mathscr{P}^{\dag} = \mathscr{P}^{-1}
\end{equation}
Si osserva che se un sistema fisico si trova un un autostato di parità
\begin{equation}
\langle\vec{x} | \psi_{P}\rangle = \pm \langle\vec{x} | \psi\rangle
\end{equation}
allora la funzione d'onda \`e una funzione pari o dispari delle coordinate:
\begin{equation}
\langle - \vec{x} | \psi\rangle = \pm \langle\vec{x} | \psi\rangle
\end{equation}
Ricorrendo alla usuale espansione della funzione d'onda in armoniche sferiche è possibile stabilire che le propriet\`a di trasformazione dipendono dallo stato di momento angolare orbitale, essendo:
\begin{equation}
Y^l_m\left( -\vec{x} \right) = \left( -1 \right)^l \cdot Y^l_m\left( \vec{x} \right)
\end{equation}
Si può quindi concludere che lo stato di parità di un sistema di $N$ particelle è dato da: 
\begin{equation}
\eta _P = \left( -1 \right)^l \prod_i^N \eta _P\left(i)\right.  
\end{equation}
dove con $\eta _P\left(i\right) $ è stata indicata la parità intrinseca della $i$-esima particella \cite{Sozzi}.

\section{Coniugazione di carica}
\noindent
La \emph{coniugazione di carica} è la simmetria che si ha sotto lo scambio particella-antiparticella. Poiché una delle conseguenze del
Teorema CPT è che ciascuna particella ha proprietà elettromagnetiche opposte a quelle della rispettiva antiparticella, spesso si cerca un 
corrispettivo classico di questa simmetria nell'invarianza delle Equazioni di Maxwell per un cambio di segno della carica elettrica.
In realtà lo scambio particella-antiparticella implica un cambio di segno di tutti i numeri quantici additivi (numero barionico, numero leptonico, etc.) 
e non della sola carica elettrica, ragion per cui la coniugazione di carica non ha un vero e proprio analogo classico\cite{Sozzi}.

\subsection{L'operatore $\mathscr{C}$} 
\noindent
Nella teoria quantistica la trasformazione di coniugazione di carica è ottenuta mediante un operatore unitario $\mathscr{C}$ sotto la cui azione l'operatore di 
carica $Q$ cambia segno:
\begin{equation}
Q \rightarrow Q_C =\mathscr{C}^{\dag} Q \mathscr{C} = -Q
\end{equation}
o, equivalentemente:
\begin{equation}
\left\{Q,\mathscr{C} \right \} = 0
\end{equation}

In meccanica quantistica relativistica (con creazione e distruzione di particelle) la carica \`e associata a una trasformazione unitaria, che per una particella di momento $\vec{p}$, spin $\vec{s}$ e carica $q$ fornisce:
\begin{equation}
\mathscr{C} | \vec{p},s,q \rangle = \eta_C \cdot \mathscr{C} | \vec{p},s,-q \rangle
\end{equation}

Per particelle rappresentate da campi fermionici (quark e leptoni) l'azione dell'operatore di coniugazione di carica può essere definito a partire dalla equazione di 
Dirac (per una derivazione di questa equazione, si veda l' \emph{Appendice A}). Per un elettrone si scrive:
\begin{equation}
 \Big[\Big(i\frac{\partial}{\partial x^\mu}-eA_\mu\Big) \gamma^\mu - m\Big] \Psi_e = 0
\end{equation}
mentre per la sua antiparticella, il positrone, si ha:
\begin{equation}
 \Big[\Big(i\frac{\partial}{\partial x^\mu}+eA_\mu\Big) \gamma^\mu - m\Big] \Psi_p = 0
\end{equation}
L'operatore $\mathscr{C}$ deve dunque trasformare le due equazioni l'una nell'altra.
Si considera la complessa coniugata dell'equazione di Dirac per l'elettrone e la si moltiplica per $-1$.
Si sfruttano le relazioni:
\begin{equation}
 i\frac{\partial}{\partial x^\mu} = -\Big(i\frac{\partial}{\partial x^\mu}\Big)^*
\end{equation}
\begin{equation}
A_\mu = (A_\mu)^*  
\end{equation}
e si ottiene:
\begin{equation}
 \Big[\Big(i\frac{\partial}{\partial x^\mu}+eA_\mu\Big) (\gamma^\mu)^* - m\Big] \Psi_e = 0
\end{equation}

La ricerca di un operatore $\mathscr{C}$ che rappresenti l'operazione di coniugazione di carica si riduce perciò a quella di una matrice non singolare $C\gamma^0$ tale che:
\begin{equation} \label{mat}
 (C\gamma^0)(\gamma^\mu)^*(C\gamma^0)^{-1} = -\gamma^\mu
\end{equation}
In questo modo si otterrà:
\begin{equation}
 (C\gamma^0)\Bigg[\Bigg(i\frac{\partial}{\partial x^\mu} +eA_\mu\Bigg)(\gamma^\mu)^* + m \Bigg] (C\gamma^0)^{-1}(C\gamma^0)\psi^* = 0
\end{equation}
\begin{equation}
  \Bigg[-\Bigg(i\frac{\partial}{\partial x^\mu} +eA_\mu\Bigg)\gamma^\mu + m \Bigg](C\gamma^0)\psi^* = 0
\end{equation}
\begin{equation}
  \Bigg[\Bigg(i\frac{\partial}{\partial x^\mu} + eA_\mu\Bigg)\gamma^\mu - m \Bigg](C\gamma^0)\psi^* = 0
\end{equation}
Si costruisce ora la matrice $(C\gamma^0)$ nella rappresentazione in cui:
\begin{equation}
 \gamma^0 = \begin{bmatrix} 1 & 0 & 0 & 0 \\ 0 & 1 & 0 & 0 \\ 0 & 0 & -1 & 0 \\ 0 & 0 & 0 & -1 \end{bmatrix}; \; \; \; \gamma^1 = \begin{bmatrix}   0 & 0 & 0 & 1\\  0 & 0 & 1 & 0  \\0 & -1 & 0 & 0\\-1 & 0 & 0 & 0 \end{bmatrix}
\end{equation}
\begin{equation}
 \gamma^2 = \begin{bmatrix} 0 & 0 & 0 & -i\\  0 & 0 & i & 0  \\0 & i & 0 & 0\\-i & 0 & 0 & 0 \end{bmatrix}; \; \; \; \gamma^3 = \begin{bmatrix}   0 & 0 & 1 & 0\\ 0 & 0 & 0 & -1 \\ -1 & 0 & 0 & 0 \\ 0 & 1 & 0 & 0 \end{bmatrix}
\end{equation}
\begin{equation}
 g = \begin{bmatrix} 1 & 0 & 0 & 0 \\ 0 & -1 & 0 & 0 \\ 0 & 0 & -1 & 0 \\ 0 & 0 & 0 & -1 \end{bmatrix}
\end{equation}
Dalle proprietà delle matrici esposte sopra, si ricava che, per tutti gli indici $\mu$, la relazione \eqref{mat} si riduce a:
\begin{equation}
 C(\gamma^{\mu})^T C^{-1} = - \gamma^{\mu}
\end{equation}
\begin{equation}
 C(\gamma^{\mu})^T = -\gamma^{\mu} C
\end{equation}
Che equivale a dire:
\begin{displaymath}
\left\{
\begin{array}{l}
 C\gamma^{\mu} = -\gamma^{\mu} C \ \ \ \ \ \ \ \ \ \ \mu = 0; 2 \\
 C\gamma^{\mu} =  \gamma^{\mu} C \ \ \ \ \ \ \ \ \ \ \mu = 1; 3
\end{array}
\right.
\end{displaymath}
Una possibile scelta di $C$, compatibile con quanto detto finora, pu\`o essere:
\begin{equation} \label{matriceC}
 C = i\gamma^2\gamma^0
\end{equation}
Si può mostrare, per calcolo diretto, che l'operatore di coniugazione di carica, definito da $\mathscr{C} = C\gamma^0$, con $C$ data da \eqref{matriceC}, gode 
delle seguenti proprietà:
\begin{equation}
 C^2 = \mathscr{I} \ \ \ \ \ \ \ C^{-1} = C^{\dag} = C
\end{equation}
Per cui anche $\mathscr{C}$ risulta hermitiano ed unitario\cite{BigiSanda}.
%
\section{Inversione temporale}
\noindent
In meccanica classica la simmetria di inversione temporale di un sistema implica l'invarianza della dinamica microscopica sotto la trasformazione:
\begin{equation}
t' \rightarrow -t
\end{equation}
 Ciò  è dovuto al fatto che la seconda legge di Newton 
\begin{equation}
\vec{F} = m \frac{d^2\vec{x}}{dt^2}
\end{equation}
contiene la derivata del secondo ordine nel tempo. 
Per cui se $\vec{x}(t)$ è una soluzione delle equazioni del moto, lo sar\`a anche $\vec{x}(-t)$. 

La velocità per inversione temporale si trasforma come:
\begin{displaymath}
\left\{
\begin{array}{l}
 t' \rightarrow -t\\
 \vec{\dot{x}}(t') = \frac{d\vec{x}(t')}{dt'} = \frac{d\vec{x}(-t)}{d(-t)} = - \vec{\dot{x}}(t) 
\end{array}
\right.
\end{displaymath}
Ciò porta a definire la trasformazione di inversione temporale come la simmetria che permette di riportare alle condizioni iniziali un qualsiasi sistema dinamico, invertendo ad un dato istante il verso dei vettori velocit\`a di tutte le particelle che lo compongono e lasciandolo evolvere senza ulteriori interventi.
%
\subsection{Operatori antilineari e antiunitari}
\noindent
Operatori lineari ed antilineari differiscono tra loro per il diverso comportamento che hanno quando si applicano a combinazioni lineari di ket.
In particolare, considerati due generici operatori $\mathscr{L}$(lineare) e $\mathscr{A}$(antilineare), valgono le relazioni:
\begin{displaymath}
\left\{
\begin{array}{l}
\mathscr{L}(\alpha |a \rangle + \beta |b \rangle) = \alpha \mathscr{L}|a \rangle  + \beta \mathscr{L}|a \rangle\\
\mathscr{A}(\alpha |a \rangle + \beta |b \rangle) = \alpha^*\mathscr{L}|a \rangle  + \beta^* \mathscr{L}|a \rangle
\end{array}
\right.
\end{displaymath}
Si osserva che un operatore antilineare non commuta con una costante, interpretata come operatore moltiplicativo, se questa ha parte immaginaria non nulla:
\begin{equation}
\mathscr{A} \cdot c = c^* \cdot \mathscr{A} \ \  \longrightarrow \
\ [\mathscr{A},c] = 2 \Im\{c\}
\end{equation}
Gli operatori antiunitari rappresentano una particolare classe di operatori unitari. 
Essi possono essere definiti come il risultato dell'applicazione successiva di due operatori:
\begin{equation}
\mathscr{O} = \mathscr{U}\mathscr{K}
\end {equation}
dove $\mathscr{U}$ è un opportuno operatore unitario, mentre $\mathscr{K}$ è l'operatore di complessa coniugazione.
\begin{equation}
\mathscr{K} (\alpha |a\rangle + \beta |b\rangle) =
\alpha^*\mathscr{K}  |a\rangle + \beta^* \mathscr{K}|b\rangle
\end {equation}
In generale, in uno spazio di Hilbert, l'operatore di complessa coniugazione non può essere definito in maniera indipendente dalla base scelta.
Per chiarire questo punto si considerano due generiche basi di ket $\{|a_k \rangle \}$ e $\{|b_j \rangle \}$.
Nella base $\{|a_k \rangle \}$, il vettore k-esimo viene rappresentato come un vettore colonna avente k-esima componente uguale ad uno e tutte 
le altre pari a zero. Ad esempio:
\begin{equation}
a_1 =
   \begin{bmatrix}
 1\\
 0\\
 0\\
\vdots
\end{bmatrix}
\end{equation}
Poiché $a_1$ è reale, l'applicazione dell'operatore $\mathscr{K}$ non provoca ad esso alcuna modifica.
\begin{equation}
|a_1 \rangle = \mathscr{K}|a_1\rangle
\end{equation}
Consideriamo lo stesso problema nella base {$b_j$}. Ora sono i ket della nuova base ad essere rappresentati come vettori colonna
aventi una componente unitaria e le altre nulle. Gli $|a_k\rangle$ possono essere da questi calcolati nel modo seguente:
\begin {equation}
|a_k \rangle= \sum {\langle b_j | a_k \rangle |b_j\rangle}
\end {equation}
Se, a questo punto, si calcolano $|a_1\rangle$ e
$\mathscr{K}|a_1\rangle$ si ottiene:
\begin{equation}
|a_1 \rangle= \sum {\langle b_j | a_1 \rangle |b_j\rangle}
\end{equation}
\begin{equation}
\mathscr{K}|a_1 \rangle= \sum {\langle b_j | a_1 \rangle^*|b_j\rangle}
\end{equation}
Nella seconda equazione si è tenuto conto che nella base impiegata tutti i $|b_j\rangle$ sono reali, quindi invarianti in seguito
all'applicazione dell'operatore di complessa coniugazione. Poiché in generale $\langle b_j | a_1 \rangle \ \neq \langle b_j | a_1
\rangle^*$, si deduce che nella nuova base $\{ |b_k\rangle \}$, $a_1$ non è più invariante per l'applicazione di $\mathscr{K}$
cioè, più in generale, che l'azione dell'operatore $\mathscr{K}$ su un dato vettore dipende dalla base scelta per rappresentare il vettore stesso.

L'operatore di inversione temporale $\mathscr{T}$, come mostreremo nella sezione seguente, è un operatore antiunitario.  Esprimendo $\mathscr{T}$ come prodotto di un operatore unitario e l'operatore di complessa coniugazione $\mathscr{K}$ si ha:
\begin{equation}
 \mathscr{T} = \mathscr{U}_T\mathscr{K}
\end{equation}
da cui si deduce che poich\'e ad esso corrisponde ad una trasformazione del sistema fisico, $\mathscr{T}$ deve poter essere definito in maniera indipendente dalla base scelta per la rappresentazione. L'operatore $\mathscr{U}_T$ 
allora a sua volta deve presentare una dipendenza dalla base che compensi quella di $\mathscr{K}$\cite{Branco}.
%
\subsection{L'operatore di inversione temporale $\mathscr{T}$ come operatore antiunitario.}
\noindent
L'operatore $\mathscr{T}$ deve agire sugli osservabili in maniera tale che questi si trasformino in maniera consistente con  quanto avviene in fisica classica. Devono valere quindi le seguenti relazioni:
\begin{equation}
\mathscr{T}^{-1} \vec{x} \mathscr{T} = \vec{x}
\end{equation}
\begin{equation}
\mathscr{T}^{-1} \vec{p} \mathscr{T} = -\vec{p}
\end{equation}
\begin{equation}
\mathscr{T}^{-1} \vec{J} \mathscr{T} = -\vec{J}
\end{equation}
Queste relazioni forniscono un primo argomento a sostegno dell'ipotesi che $\mathscr{T}$, a differenza di $\mathscr{C}$ e $\mathscr{P}$, non pu\`o essere un operatore unitario, ma deve essere piuttosto un operatore antiunitario. \\
Si consideri la relazione di commutazione tra l'operatore coordinate e l'operatore impulso:
\begin{equation}
[x_j, p_k] = i\delta_{jk}
\end{equation}
$\mathscr{T}$ modifica il segno di $p_k$ ma non quello di $x_j$, per cui il calcolo del commutatore in seguito all'applicazione della riflessione temporale risulta consistente con le regole di commutazione quantistiche solo se si accetta l'ipotesi che
$i\delta_{jk}$ cambi segno in seguito alla trasformazione, cio\`e che $\mathscr {T}$ sia antiunitario.\\
Si consideri adesso l'applicazione di $\mathscr{T}$  ad un sistema fisico in evoluzione.
L'operatore di evoluzione temporale S \`e definito come:
\begin{equation}
|\psi(t+dt)\rangle = S|\psi(t)\rangle = (1-\frac{iH}{\hbar}dt)|\psi(t)\rangle
\end{equation}
dove $\psi$ è la funzione d'onda del sistema, $H$ l'operatore hamiltoniano ed $1$ l'identità.

Si consideri quindi l'applicazione dell'operatore di evoluzione ad un sistema a cui è stato applicato $\mathscr{T}$; si consideri poi l'applicazione di $\mathscr{T}$ a un sistema lasciato  libero di evolvere indietro nel tempo, per un intervallo di tempo $d(-t)$.
Poich\'e si stanno trattando sistemi per i quali T rappresenta una simmetria esatta, queste operazioni devono portare allo stesso risultato netto. Per cui deve valere l'uguaglianza:
\begin{equation}
 (1-\frac{iH}{\hbar}dt)\mathscr{T}|\psi(t)\rangle = \mathscr{T}(1-\frac{iH}{\hbar}d(-t))|\psi(t)\rangle
\end{equation}
da cui si arriva a:
\begin{equation}\label{equazio}
 -iH\mathscr{T}|\psi(t)\rangle =  \mathscr{T}iH|\psi(t)\rangle
\end{equation}
Imponendo la relazione di commutazione dell'operatore d'inversione temporale con l'Hamiltoniana del sistema $[\mathscr{T},\mathscr{H}] = 0$, e considerando che $|\psi(t)\rangle$ \`e una funzione d'onda arbitraria, si conclude che la relazione \eqref{equazio}, pu\`o essere soddisfatta se solo se $\mathscr{T}$ \`e un operatore antiunitario\cite{BigiSanda}.

\subsection{Autovalori di $\mathscr{T}$,$\mathscr{T}^2$ e degenerazione di Kramer}
\noindent
Contrariamente a ciò che avviene per gli operatori $\mathscr{C}$ e $\mathscr{P}$, gli autovalori di $\mathscr{T}$ non hanno significato fisico, quindi non \`e possibile associare un numero quantico conservato a questo operatore.
Si consideri a questo scopo l'applicazione di $\mathscr{T}$ ad una generica funzione d'onda, all'istante $t = 0$:
\begin {equation}
\mathscr{T}|\psi(0)\rangle = c|\psi(0)\rangle
\end {equation}
dove $c$ rappresenta l'autovalore relativo a $\psi(0)$. 
Si lascia dunque evolvere il sistema e si considera:
\begin{equation}
|\psi(dt)\rangle = S |\psi(0) \rangle
\end{equation}
Applicando l'inversione temporale ai due lati di questa equazione si ottiene:
\begin{equation}
\mathscr{T}|\psi(t)\rangle = \mathscr{T}S|\psi(0)\rangle =
\mathscr{T}S\mathscr{T}^{-1}\mathscr{T}|\psi(0)\rangle = cS^{\dag}|\psi(0)\rangle
\end{equation}
Nell'ultima uguaglianza si \`e sfruttata la relazione che segue dall'antiunitariet\`a di $\mathscr{T}$, 
$S^{\dag} = \mathscr{T} S {\mathscr{T}}^{-1}$.\\
In generale, $cS^{\dag}|\psi(0)\rangle$ non \`e proporzionale a $|\psi(t)\rangle$,
per questo motivo non \`e possibile associare la conservazione di un numero quantico agli autovalori di $\mathscr{T}$.\\
Il ragionamento fin qui esposto si fonda sull'ipotesi di antiunitariet\`a di $\mathscr{T}$.
L'operatore di doppia inversione temporale $\mathscr{T}^2$, d'altro canto, \`e unitario e quindi ad esso \`e possibile associare autovalori di significato fisico.
L'applicazione di $\mathscr{T}^2$ pu\`o modificare la funzione d'onda del sistema al pi\`u per un fattore di fase:
\begin{equation}\label{equazione}
\mathscr{T}^2 = \mathscr{U}\mathscr{K}\mathscr{U}\mathscr{K} =
\mathscr{U}\mathscr{U}^*\mathscr{K}\mathscr{K} =
\mathscr{U}\mathscr{U}^* = \eta
\end{equation}
dove $\eta$ \`e appunto una matrice diagonale di elementi complessi di modulo unitario, che indicheremo con $(e^{i\eta_1} ...\ e^{i\eta_n})$.
Dalla definizione di operatore unitario segue che $U^*=(U^T)^{-1}$, quindi dall'ultima uguaglianza della \eqref{equazione} si ottiene:
\begin{equation}
U = \eta U^T
\end{equation}
trasponendo a destra e a sinistra:
\begin {equation}
U^T = U\eta
\end{equation}
infine, sostituendo nella precedente:
\begin{equation}
U = \eta U \eta
\end{equation}
Per ogni coppia di indici $j$ e $k$ si deve avere:
\begin{equation}
 e^{i(\eta_j + \eta_k)} = 1
\end{equation}
Quindi si pone $j = k$ e si ottiene:
\begin{equation}
 e^{i\eta_i} = \pm1
\end{equation}
Si è così dimostrato che l'operatore di doppia inversione temporale ammette come soli possibili autovalori $+1$ e $-1$.
È possibile dimostrare che i sistemi fermionici sono autostati di $\mathscr{T}^2$ di autovalore $-1$, i sistemi bosonici sono autostati di autovalore $+1$.
Tutto ciò ha un'importante conseguenza: in un sistema fermionico invariante rispetto a $\mathscr{T}$, gli autostati dell'energia ammettono una (almeno) doppia degenerazione.
La dimostrazione è molto semplice. Si consideri un autostato dell'energia $|E\rangle$ di autovalore $E$:
\begin{equation}
 H|E\rangle = E|E\rangle
\end{equation}
Sia ora $|E_{\mathscr{T}}\rangle$ il trasformato di $|E\rangle$ sotto $\mathscr{T}$. Per l'invarianza del sistema anche questo autostato deve avere autovalore $E$.
Inoltre, ricordando che $\mathscr{T}^2|E\rangle = -|E\rangle$ si ricava:
\begin{equation}
 \langle E|E_{\mathscr{T}}\rangle = \langle E_{\mathscr{T}}|\mathscr{T}^{\dag}\mathscr{T}|E\rangle = \langle E|(\mathscr{T}^{\dag})^2\mathscr{T}|E\rangle = -\langle E|\mathscr{T}|E\rangle = \langle E|E_{\mathscr{T}}\rangle
\end{equation}
Per cui:
\begin{equation}
 \langle E|E_{\mathscr{T}}\rangle = 0
\end{equation}
I due autostati sono tra loro ortogonali, quindi non rappresentano lo stesso stato fisico. Quanto qui dimostrato viene chiamato \emph{degenerazione di Kramer}\cite{BigiSanda}.