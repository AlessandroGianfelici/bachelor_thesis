     %%%%%%%%%%%%%%%%%%%%
     %                  %
     %  conclusioni.tex %
     %                  %
     %%%%%%%%%%%%%%%%%%%%

\chapter*{Conclusioni}
\noindent
Nella tesi si è proposta una discussione articolata del fenomeno della violazione di CP nelle interazioni deboli. 

Dapprima si è data la definizione delle tre simmetrie discrete e dei relativi operatori, per poi concentrarsi su CP. La violazione di quest'ultima è stata discussa 
sul piano teorico, quindi è stato sviluppato il formalismo matematico adatto a trattare la dinamica di un sistema di mesoni neutri $P^0$ - $\bar{P}^0$.

Successivamente si è presentato il quadro delle interazioni deboli nel Modello Standard. In quest'ambito è stata sottolineata l'importanza della misura  
dei parametri della matrice CKM, in particolare delle grandezze caratteristiche del triangolo di unitarietà ad essa associato:
\begin{equation*}
 V_{ud}V_{ub}^* + V_{cd}V_{cb}^* + V_{td}V_{tb}^* = 0
\end{equation*}
Si è quindi proposta la misura dell'angolo $\gamma$ connesso a questo triangolo, definito dall'espressione:
\begin{equation*}
 \gamma = \arg\Big(-\frac{V_{ud}V_{ub}^*}{V_{cd}V_{cb}^*}\Big).
\end{equation*}

Per realizzare questa misura sono stati combinati i risultati dei parametri sensibili a $\gamma$ ottenuti da LHCb  mediante la rivelazione dei decadimenti del 
$B^{\pm}\rightarrow DK^{\mp}$ durante il 2011, corrispondenti ad una luminosità di $1$ $fb^{-1}$.

I metodi utilizzati per l'analisi sono GLW, ADS e GGSZ. Combinati tra loro, essi portano a definire un sistema di equazioni fortemente non lineari, che legano
le osservabili misurate attraverso opportuni decadimenti $B^{\pm}\rightarrow DK^+$ a $\gamma$. L'analisi è stata realizzata utilizzando il metodo statistico bayesiano,
mediante un \emph{software} chiamato BAT (\emph{Bayesian Analysis Toolkit}).

È stata così ottenuta la distribuzione statistica \emph{a posteriori} per il valore di $\gamma$ e, a partire da questa, gli intervalli di confidenza al $68\%$ ed al $95\%$.
Il risultato finale ottenuto è:
\begin{equation*}
\gamma = (60.3_{-13.5}^{+17.1})^{\circ} 
\end{equation*}

Le Collaborazioni BaBar\cite{BaBar} e Belle\cite{Belle} hanno misurato di recente $\gamma$ dai decadimenti $B^{\pm}\rightarrow K^{\pm}$ ottenendo i valori
$\gamma_{BaBar} = 69^{+17}_{-16}$ e $ \gamma_{Belle} =  68^{+15}_{-14}$. Il risultato ottenuto in questa tesi risulta quindi in ottimo accordo con essi, il che dimostra 
il grande potenziale di LHCb, considerando che l'analisi è stata condotta con $1$ $fb^{-1}$ di luminosità integrata, pari a circa il $40\%$ della statistica totale raccolta da 
LHCb ad oggi (LHCb ha infatti raccolto in totale $2.5$ $fb^{-1}$).


